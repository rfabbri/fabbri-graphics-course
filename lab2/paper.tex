\documentclass[a4paper]{article}
\usepackage[utf8]{inputenc}

\usepackage{graphicx}
\usepackage[bf]{subfigure}
\usepackage[bf]{caption}
% Section titles in sans-serif:
\usepackage{sectsty}
\usepackage{abstract}
\allsectionsfont{\sffamily}

%\renewcommand{\abstractname}{\textsf{Abstract} xxxx}
\renewcommand{\abstractnamefont}{\bfseries\textsf}

\usepackage[usenames,dvipsnames]{color}
\definecolor{seagreen}{RGB}{46,139,87}
\definecolor{codebg}{RGB}{255,255,150}
\usepackage{listings}
\usepackage[colorlinks]{hyperref}
\usepackage{url}	

\hypersetup{%
  citecolor=blue,
  linkcolor=blue,
  urlcolor=blue,
}%



\lstset{
  extendedchars=\true,
  inputencoding=utf8,
  language=Matlab,
  %showstringspaces=false,
  formfeed=\newpage,
  tabsize=4,
  %commentstyle=\itshape,
  basicstyle=\ttfamily\scriptsize,
  %basicstyle={\small\fontfamily{fvm}\fontseries{m}\selectfont},
  commentstyle=\color{Apricot}\bfseries,
  %commentstyle=\color{red}\itshape,
  stringstyle=\color{red},
  identifierstyle=\color{PineGreen},
  showstringspaces=false,
  keywordstyle=\color{blue}\bfseries,
  moredelim=[il][\large\textbf]{\#\# },
  morekeywords={models,range},
  numbers=left,
  numbersep=2pt,
  numberstyle=\tiny,%\color{blue}\bfseries,
  backgroundcolor=\color{codebg},
  literate=%
  {ã}{{\~a}}1
  {â}{{\^a}}1
  {õ}{{\~o}}1
  {á}{{\'a}}1
  {ú}{{\'u}}1
  {í}{{\'i}}1
  {é}{{\'e}}1
  {Ç}{{\c{C}}}1
  {Õ}{{\~O}}1
  {Ê}{{\^E}}1
  {ó}{{\'o}}1
  {à}{{\`a}}1
  {Â}{{\^A}}1
  {ô}{{\^o}}1
  {ê}{{\^e}}1
  {ç}{{\c{c}}}1
}

\newcommand{\code}[2]{
 \vspace{1em}
 \subsubsection*{#1}
 \lstinputlisting{#2}
}

%%%%%%%%%%%%%%%%%%%%%%%%%%%%%%%%%%%%%%%%
% You have two versions of the macro
% \draftnote{My note}. The first version puts notes (e.g. My note in the example)
% into the margin of your document. The second formats the note as nothing. You
% 'comment out' the version of the macro you don't want (using a % at the
% beginning of the line).
\newcommand{\draftnote}[1]{\marginpar{\tiny\raggedright\textsf{\hspace{0pt}#1}}}
%\newcommand{\draftnote}[1]{}

% This one is just for the comments for in-line text.
\newcommand{\indraftnote}[1]{\textcolor[HTML]{114406}{\texttt{\footnotesize[#1]}}}
%\newcommand{\indraftnote}[1]{}
\newcommand{\todo}[1]{\indraftnote{todo: #1}}
\newcommand{\ie}{{\it i.e.}}
\newcommand{\etc}{{\it etc}}
\newcommand{\eg}{{\it e.g.}}
\newcommand{\wrt}{{\it w.r.t. }}
\newcommand{\etal}{{\it et.\ al.\ }}
\newcommand{\etalf}{{\it et.\ al.}}


\begin{document}

\title{\textsf{Computer Graphics Lab \#2\\ Splines, Animation and 3D}
\marginpar{\vspace{-1.2cm}\includegraphics[height=1.4cm] {figs/logo_uerj_cor_small.png}}} 

\author{Prof.\ Ricardo Fabbri, Ph.D.\\[1em]
\small{Polytechnic Institute (IPRJ) at the Rio de Janeiro State University}\\
\small{\url{http://wiki.nosdigitais.teia.org.br/CG}}
}
 

\date{\today}
\maketitle
\begin{abstract}
Now that you finished Lab\#1 which consisted of rotating a camera around an
arbitrary axis using the Blender Python API, its time to move on to further concepts and consolidate your
learning. This should help you with the bigger goal of the final project
of generating a complex animation.
\end{abstract}
\vspace{2em}

Any material need for the lab can be found from the course
website. Please render all your videos in full HD 1080p quality.
\textbf{All extra work will be considered for bonus grade points.}

\section{Construct the IPRJ Logo in 2D}
Your goal is to generate the IPRJ logo of Figure~\ref{fig:iprj:logo} (or any
official variant) in Blender~\cite{blender},
using two Python API's: Blender OpenGL \texttt{bgl} and standard Blender \texttt{bpy}. 
\begin{figure}
\centering
\includegraphics[width=0.5\linewidth]{figs/logo-iprj2.png}%
\caption{% 
IPRJ logo, to be programmatically drawn with splines and animated, in both
Blender and OpenGL.
}\label{fig:iprj:logo}
\end{figure}


Steps:
\begin{enumerate}
\item Construct the IPRJ logo in 2D using the standard Blender API
(\texttt{bpy}), without any
data file. The standard blender API is used to help you construct models that
can be further manipulated by other Blender scripts or GUI. Use 
spline curves and surfaces for the sine wave and curved letters, possibly for everything.
\item Construct the IPRJ logo in 2D using the Blender OpenGL API
(\texttt{bgl}), without any
data file. The OpenGL API is best for making blender tools that display
interactive graphics
\emph{prior} to a full render. Use 
spline curves and surfaces for the sine wave and curved letters, possibly for everything.
\end{enumerate}

\section{Animate the IPRJ Logo}
\begin{enumerate}
\item Create an animation of the IPRJ logo. Make the sine wave oscilate in 2D or 
3D (off the plane). Do this with Blender OpenGL and standard Blender API
\item Make the ball bounce or roll in a physically pleasing way. You may choose
to do it in \texttt{bgl} or \texttt{bpy} API (and similarly for the next items)
\item Make an intense zoom-in effect in some part and move the zoom around --
you will realize an advantage of
using splines instead of polylines: \emph{scalability}!
\item Devise your own effects. Make the sine wave oscilate in crazy ways,
\eg, modulate it a real sound signal. Add a motion blurring effect, brushed
metal textures, gradients, multiple illumination, futuristic glass texture, lens
flare, write your own shader in GLSL or Cg language. Be creative!
\item \textbf{If done with
stunning, professional high quality, or contains a great many different ways of
doing things, or lasts more than 5min, this can also pass as a final project.}
You may use the blender GUI for some or most of this item, you don't have to do everything programmatically.
\end{enumerate}

\bibliographystyle{IEEEtran}
\bibliography{personal}

\end{document}
